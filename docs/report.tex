%!TEX TS-program = xelatex
%!TEX encoding = UTF-8 Unicode

\title{
    \Large Fish-eye transformation with bilinear interpolation \\
    \large CS-425 Program parallelization on PC clusters}
\author{
    Yoan Blanc \texttt{yoan.blanc@epfl.ch}
}
\date{\today}

\documentclass[10pt,a4paper]{article}
\usepackage[svgnames]{xcolor}
\usepackage{graphicx}
\usepackage{hyperref}
\usepackage{url}
\usepackage{verbatim}
\usepackage{fontspec}
\usepackage{tikz}
\usepackage{framed}
\setmainfont[Mapping=tex-text,Ligatures={Common,Rare,Discretionary}]{Linux Libertine O}
\newfontfamily\quotefont[Mapping=tex-text,Ligatures={Common,Rare,Discretionary}]{Linux Libertine O}

% http://tex.stackexchange.com/questions/16964/block-quote-with-big-quotation-marks
% Make commands for the quotes
\newcommand*{\openquote}{\tikz[remember picture,overlay,xshift=-15pt,yshift=-10pt]
\node (OQ) {\quotefont\fontsize{60}{60}\selectfont``};\kern0pt}
\newcommand*{\closequote}{\tikz[remember picture,overlay,xshift=15pt,yshift=10pt]
\node (CQ) {\quotefont\fontsize{60}{60}\selectfont''};}
% select a colour for the shading
\definecolor{shadecolor}{named}{Azure}
% wrap everything in its own environment
\newenvironment{shadequote}%
{\begin{snugshade}\begin{quote}\openquote}
{\hfill\closequote\end{quote}\end{snugshade}}

\begin{document}
\maketitle

\section{Introduction}

This report will explore a fish-eye transformation with a bilinear
interpolation running in serial mode (one process) and how it’ll be
parallelized using either message passing (MPI, DPS) or shared memory
alternatives (OpenMP). At this point, no parallel implementation have been
done.

\section{Analysis}
\textbf{TODO}

\section{Implementation}
\textbf{TODO}

\subsection{Serial}
\textbf{TODO}

\subsection{Shared memory}
\textbf{TODO}

\subsection{Message passing}
\textbf{TODO}

\subsection{Benchmarks}
\textbf{TODO}

\section{Conclusion}

\subsection{Algorithms}

\begin{shadequote}
    FIRST OPTIMIZE ON ONE CORE, THEN PARALLELIZE (the right
    algorithm) \\ ---\emph{Vincent Keller}
\end{shadequote}

This section will present the one core (serial) algorithms that may be
parallelized. Any of them, tries to tackle a particular problem of its
predecessor(s): computational time, I/O time or memory usage.

\begin{itemize}
    \item \textbf{Serial 0}\\
    One naive version using ``ray tracing'' (going from destination to source)

    \item \textbf{Serial 1} ~(computational)\\
    \emph{Serial 0} + the naive version with pre-calculated rays optimization.

    \item \textbf{Serial 2} ~(computational)\\
    \emph{Serial 1} + the pre-calculated optimization for the fish-eye (lens)
    zone only.

    \item \textbf{Serial 3} ~(memory, computational)\\
    \emph{Serial 2} + making the transformation in place plus some
    computation optimizations (\verb|for-loops|)

    \item \textbf{Serial 4} ~(memory)\\
    \emph{Serial 3} + use a smaller mask, only $\frac{1}{4}$ of the full lens.

    \item \textbf{Serial 5} ~(I/O)\\
    \emph{Serial 4} + Opening the source file using \texttt{mmap}
    (\verb|MMAP(2)|) and not working in-place anymore.

    \item \textbf{Serial 6} ~(memory, I/O)\\
    \emph{Serial 4} + Copying the source file to destination using
    \texttt{sendfile} (\verb|SENDFILE(2)|), working in place and using
    \texttt{mmap} \end{itemize}.

\subsubsection{Serial 0}

The naive serial version works as follow:

\begin{itemize}
    \item each destination pixel $(x,y)$ is converted into its polar
    coordinates equivalent. The origin $(0,0)$ is the center of the picture
    $(\frac{width}{2},\frac{height}{2})$.
    
    \item now it’s source point is calculated (the solid arrow). If the point
    $(r,\theta)$ is:

    \begin{itemize}
        \item outside the fisheye radius, the source pixel is the same as the
        destination pixel.


        \item inside the fisheye radius, the source pixel
        $(r\prime,\theta\prime)$ is calculated using the formula given in
        the project description ($m$ is the magnification factor):
        
        $ \theta\prime = \theta $ \\
        $ r' = \frac{1}{1+m\frac{1-r}{r}} $
        
    \end{itemize}

    \item then a bilinear interpolation is made between $(x\prime,y\prime)$
    which may be real number and the $4$ surrounding pixels in the source
    picture (the dotted arrow).

\end{itemize}

\subsubsection{Serial 1}

This improvement does the very same as serial 0 but will make $25\%$ less polar
to radius calculation by pre-calculating the $(x\prime,y\prime)$ values for
each pixel. The fisheye mask as a central symmetry. So it exploits it to only
calculate the top left quarter of the picture and then transpose it to the $3$
others. Now while building the destination picture, the source pixel is
obtained by using the mask (the solid arrow) instead of calculating everything
(the dashed arrow).  The billinear interpolation (the dotted arrow) still needs
to be done since there are no assumed symmetric in the source picture.

\subsubsection{Serial 2}

It’s building upon \emph{Serial 1}, with a little improvement. The mask is not
only symmetrical in $4$ but in $8$ as well easily if the mask is a square. And
as everything outside the fisheye lense is set to null, it’s simply wasted
space. In face, the mask is infinity-symmetrical but it gets tricky in
cartesian coordinates (but doable still if the mask’s size are okay). Here it
does less calculation and uses less storage for the mask.

\subsubsection{Serial 3}

Building from \emph{Serial 2}, it removes the need of having a source and a
destination and works in the source data directly. It wasn’t possible because
while read from top to bottom and left to right, everything but the top left
corner will modify data it’ll read later on. The algorithm here will act
differently for each quarter of the output picture. And other gain, everything
from the source picture outside the lens don’t have to be copied over to the
destination one.

\subsection{Machines}

Several machines have been used for this project and any number mentions which
one was used.

\begin{itemize}
    \item \textbf{Macbook5,2}\\
    my personal laptop running Gentoo Linux, x86\_64, Intel Core 2 Duo P7550
    @2.26GHz, 2Go RAM.
    
    \item \textbf{BC}\\
    The machines from BC lab running Ubuntu Linux 10.10, x86\_64, Intel Core i7
    920 @2.66GHz, 6Go RAM.

\end{itemize}

%\bibliographystyle{unsrt}
%\bibliography{report}

\end{document}
